% Først spesifiserer vi hvilken dokumentklasse vi vil ha og noen 
% globale opsjoner. Bytt ut 'article' med 'book' hvis du vil ha 
% med kapitler.
\documentclass[a4paper, twoside, titlepage, 11pt]{article}

% SÃ¥ sier vi fra om hvilke tilleggspakker vi trenger
% til dokumentet vårt. De som du ikke trenger (se kommentaren) 
% kan det være en fordel å kommentere ut (sett prosenttegn foran),
% da vil kompilering gå raskere.

\usepackage[T1]{fontenc}		% norsk tegnsett (æøå)
\usepackage[utf8]{inputenc}	% norsk tegnsett
\usepackage{geometry}		% anbefalt pakke for å styre marger.

\usepackage{amsmath,amsfonts,amssymb} % matematikksymboler
\usepackage{amsthm}                   % for å lage teoremer og lignende.
\usepackage{graphicx}                 % inkludering av grafikk
\usepackage{subfig}                   % hvis du vil kunne ha flere
% figurer inni en figur
\usepackage{listings}                 % Fin for inkludering av kildekode
\begin{document}	
	% NÃ¥ vi vil ha noe i topp- og bunnteksten
	\pagestyle{headings}
	
	% Arabisk (vanlige tall) sidenummerering. Starter på side 1 igjen.
	\pagenumbering{arabic}
	
	\section{Neural network simplification}
	\paragraph{Simplification}
	The network can be simplified to some degree using some maths.
	
	\paragraph{Definition of output neuron}
	The output neuron is is defined as the following.
	\begin{equation}
	O = sign(B(W * S))
	\end{equation}
	Where B is the batchnorm function, W is the weight matrix, and S is the synapses (synapses are the output of the previous layer). ''sign'' is the sign function, giving -1 if the number is negative and 1 if it is positive. The batchnorm function is defined as follows.
	\begin{equation}
		B(x) = \gamma * \sigma^{-1} * (x - \bar{x}) + \beta
	\end{equation}
	\(\gamma\), \(\sigma^{-1}\), \(\bar{x}\) and \(\beta\) is provided with the network. This function will be applied on each element in the \(W * S\) vector. The following equations will show a simplification of this function.
	\begin{equation}
		B(x) = \gamma * \sigma^{-1} * (x - \bar{x}) + \beta
	\end{equation}
	\begin{equation}
		B(x) = \gamma * \sigma^{-1} * x - \gamma * \sigma^{-1} * \bar{x} + \beta
	\end{equation}
	This can be rewritten as
	\begin{equation}
		B(x) = a * x + b
	\end{equation}
	Where
	\begin{equation}
		a = \gamma * \sigma^{-1}
	\end{equation}
	and
	\begin{equation}
		b = -\gamma * \sigma^{-1} * \bar{x} + \beta
	\end{equation}
	Since we are only interested in the sign of \(B(x)\), it is possible to simplify even more. Multiplying by \(\frac{1}{a}\), will not change the sign of the expression. The multiplication might change the sign of \(a * x\), but will also change the sign of \(b\) by the same amount.
	\begin{equation}
		sign(B(x)) = sign(a * x + b) = sign((a * x + b) * \frac{1}{a}) = sign(x + \frac{b}{a})
	\end{equation}
	In the case of pre trained neural networks, the values of \(a\) and \(b\) will not change. Then the \(\frac{b}{a}\) can be precomputed, saving time when feeding forward inputs.  The sign operation of a subtractions is really just a comparison. So in this case, it is sufficient to compare the values of \(x\) and \(\frac{b}{a}\) and give an output accordingly.
	
	\paragraph{Summary}
	Each layer in the feed forward network can be simplified by the binary matrix multiplication with the previous output \(W * S\), and then comparing with a single vector to determine if the output should be 1 or -1 (Compare to \(\frac{b}{a}\)).
	
\end{document}

% Local Variables:
% TeX-master: "master"
% End:
